\documentclass[12pt,a4paper]{article}

% incluyendo paquetes
\usepackage[utf8]{inputenc}
\usepackage[spanish]{babel}
\usepackage[acronym]{glossaries}
\usepackage[table]{xcolor}
\usepackage{milibreria}
\usepackage{placeins}

\newcommand{\dpsec}{Dirección de Proyección Social y Extensión Cultural}
\newcommand{\genseg}{Genseg}
\makeglossaries
\newacronym{dpsec}{DPSEC}{Dirección de Proyección Social y Extensión Cultural}
\newacronym{dr}{Dr.}{Doctor}
\newacronym{etc}{etc.}{et cetera}
\newacronym{ieee}{IEEE}{Instituto de Ingenieros Eléctricos y Electrónicos}
\newacronym{ers}{ERS}{especificación de requisitos software}
\newacronym{genseg}{GENSEG}{Gestor de Planes y Proyectos de la Direccion de Proyeccion Social y Extension Cultural}
\newacronym{xp}{XP}{Extreme Programming}

\graphicspath{{C:/Users/david/imagesppt}} %\incluye todos las imágenes de esa ruta
%\graphicspath{{D:/proyectos_latex/7mo_semestre/gestion_redes/informe_de_redes/main/images}}
\begin{document} % inicio  de documento 

%\include{caratula.tex} % incluyendo la caratula
\usetikzlibrary{calc}
\begin{titlepage}
    \centering
    \miRectangulo{-3.2cm}{10cm}{13.7cm}{4.9cm}{azulm}{0}
    \miRectangulo{0cm}{7.5cm}{9cm}{6cm}{azulm}{45}
    \miRectangulo{-4cm}{-15cm}{15cm}{6cm}{azulm}{0}

    \miRectangulo{-16cm}{-7cm}{10cm}{6cm}{azulm}{45}
    \miRectangulo{-11cm}{-15cm}{9cm}{3cm}{azulm}{0}
    \miRectangulo{-5cm}{-4.5cm}{3.5cm}{3.5cm}{azulm}{45}
    \begin{figure}[h]
        \miRomboImagen{-2.3}{-10.5}{images/magno.jpeg}{20}
    \end{figure}
    
    \begin{textblock}{70}(135,13)
        \begin{flushright}
            {\large{\textcolor{white}{Dirección de Proyección Social y Extensión Cultural}}}\\
            %{\normalsize{\textcolor{white}{Educando mentes, Cambiando el mundo}}}
        \end{flushright} 
    \end{textblock}
    \begin{tikzpicture}[remember picture, overlay]
        \node at (current page.north west) [anchor=north west, xshift=90mm, yshift=-2mm] {\includegraphics[width=0.25\textwidth]{images/logodpsecsf.png}};
    \end{tikzpicture}

    \begin{textblock}{100}(100,60)
        \begin{flushright}
            {\Huge{\textbf{2024}}}\\[20pt]
            {\fontsize{50}{60}\selectfont\textbf{\MakeUppercase{\genseg}}}\\[20pt]
            {\large{\textbf{Gestor de Proyectos de la Dirección de Proyección Social y Extensión Cultural}}}\\[15pt]
            {\Large{\textbf{Documento de Propuesta de Desarrollo}}}\\[15pt]
        \end{flushright} 
    \end{textblock}

    \begin{textblock}{100}(120,140)
        \textcolor{azulm}{\rule{0.5\linewidth}{0.90mm}} \par
    \end{textblock}
    
    \begin{textblock}{100}(100,160)
        \begin{flushright}
            {\large{\textbf{Fecha: \\[10pt] \today}}}\\[15pt]
            {\large{\textbf{ Propuesto por: }}}\\[5pt]
            {\large{\textbf{ Larota Pilco David Bahyan }}}\\[5pt]
            {\large{\textbf{ Pari Choquehuanca Bernardo }}}\\[5pt]
            {\large{\textbf{ Quispe Bravo Marco Alexander }}}\\[5pt]
            {\large{\textbf{ Chura Cutipa Lenin Alonso }}}\\[5pt]
            %{\large{\textbf{ Apaza Ccapa Rosmery }}}\\[5pt]
        \end{flushright} 
    \end{textblock}

    \begin{textblock}{100}(100,256)
        \begin{flushright}
            {\huge{\textcolor{white}{Universidad Nacional del Altiplano}}}\\
            {\normalsize{\textcolor{white}{Educando mentes, Cambiando el mundo}}}
        
        \end{flushright} 
    \end{textblock}
    \begin{tikzpicture}[remember picture, overlay]
        \node at (current page.north west) [anchor=north west, xshift=50mm, yshift=-250mm] {\includegraphics[width=0.25\textwidth]{images/qr_code.png}};
    \end{tikzpicture}

\end{titlepage}
\tableofcontents % índice automático
% Índice de figuras
\listoffigures

% Índice de tablas
\listoftables
\pagestyle{fancy} \mystyle \newpage % Aplicar el estilo de encabezado y pie de página
% inicio del documento




\section{Introducción }
Este plan de Desarrollo de software es una versión preliminar preparada para ser incluida 
en la propuesta elaborada como respuesta el proyecto \textquotedblleft(\gls{dpsec})\textquotedblright \ Este documento
provee una visión global del enfoque de desarrollo propuesto.
\espacio
En el proyecto se usa una metodologia \textbf{\gls{xp}} en la que únicamente
se procederá a cumplir con las fases: \textbf{planificación, diseño, codificación, pruebas y lanzamiento.}
que marca la metodologia de desarrollo de software \gls{xp}.
Se incluirá el detalle para las fases de Análisis, Diseño, Desarrollo e Implementación.
del sistema propuesto para la gestion de proyectos y planes.
\espacio
El enfoque de desarrollo propuesto constituye una configuración del proceso
de gestion de proyectos y planes dentro de la \gls{dpsec} de acuerdo a las 
caracteristicas del proyecto \gls{genseg}.

\subsection{Propósito }

\subsection{Alcance}
 
\subsection{Justificación/Resumen}


\section{Personal involucrado}

\begin{table}[h!]
    \centering
    \begin{tabular}{|p{7cm}|p{8cm}|}
    \hline
    \rowcolor{pastelBlue} \textbf{Nombre} & David Brahyan Larota Pilco \\ \hline
    \textbf{Rol} & Analista, diseñador y programador \\ \hline
    \rowcolor{pastelBlue} \textbf{Categoría profesional} & Ingeniería de sistemas \\ \hline
    \textbf{Responsabilidad} & Análisis de información y diseño \\ \hline
    \rowcolor{pastelBlue} \textbf{Información de contacto} & \href{mailto:dlarotap@est.unap.edu.pe}{\textbf{dlarotap@est.unap.edu.pe *}} \\ \hline
    \end{tabular}
    \caption{Tabla de información personal}
    \label{tab:personal_info}
\end{table}




\section{Ámbito del Sistema}
\textbf{Nombre del sistema:} Sistema de recopilación de información en los proyectos de la \gls{dpsec}.
\espacio 
El sistema recopila información estructurada de las actividades, planes y proyectos de los programas de estudio (44 programas de estudio).

\espacio 
Los principales beneficios en la reducción de papel, articulación con los programas de estudio, logro de indicadores institucionales y administrativos información en tiempo real oportuna procesos eficientes justo a tiempo.

\subsection*{Las Metas principales son:}
\begin{itemize}
    \item Cobertura de 44 programas de estudio.
    \item La dirección DPSEC y las subunidades.
    \item Satisfacer a los 132 usuarios del sistema.
    \item abarcar documentos estructurados como proyectos y planes.
    
    \end{itemize}
    
    
    \subsection*{Objetivo general }
    recopilar requisitos que tengan la información para que permita implementar el sistema de Recopilación de Datos.
    
    
    \section{Definiciones, Acrónimos y Abreviaturas}
    \subsection*{Definición}
    \Dbox{Title}{hola a todos asmldnksjahnd jsahdklsa hdlsa hldhsajdskñaj }

    \printglossaries

    
    \section{Resumen }

El resumen de las 4 etapas

\begin{figure}[!htbp]
    \centering
    \includegraphics[width=0.7\textwidth]{images/magno.jpeg}
    \caption{Descripción de la imagen}
    \label{fig:miImagen}
\end{figure}
\FloatBarrier
\section{asdsa}

adasdsad mirando la tabla [tab:\ref{tab:personal_info}]
\shortcite{audit_operativa}
    \begin{comment}
    usuario
\subsubsection*{Administrador} Un super administrador que administre a otro administrador, roles por facultad.
\subsubsection*{Requisito funcional} Recopilar documento, información sobre un plan de trabajo de las 44 carreras y 3 subunidades.
\subsubsection*{Requisito no funcional} Si se quita un rol a una unidad, se quitará para todos sin excepción .


\subsection*{Acrónimos}
\begin{table}[!htbp]
    \centering
    \caption{Acrónimos}
    \label{tb:acronimos}
    \begin{tabular}{cl}
        \toprule
        \textbf{Acronimo} & \textbf{Significado} \\
        \midrule
        ERS & Especificación de requisitos de Software. \\
        UML & Lenguaje de Modelo Unificado. \\
        API & Interfaz de Programación de Aplicaciones. \\
        DPSEC & Dirección de Proyección Social y Extensión Cultural. \\
        IEEE & Práctica recomendada para especificaciones de requisitos de software. \\
        Administrador & Persona que usará el sistema para definir roles de otros administradores de menor Jerarquía. \\
        RFA & Requerimiento Funcional Administrativo.  \\

        \bottomrule
    \end{tabular}
\end{table}


\subsection*{Abreviaturas}
El \gls{dr} Smith es un destacado investigador en su campo. La lista incluye manzanas, peras, \gls{etc}.

\printglossaries
\end{comment}


\newpage
\section{Referencias}
\bibliographystyle{apacite}
\bibliography{referencias.bib}


\end{document}